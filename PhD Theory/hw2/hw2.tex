\documentclass[a4paper,12pt]{article}
\usepackage{fancyhdr}
\pagestyle{fancy}
\fancyhf{}
\lhead{Tim Vigers}
\rhead{BIOS 7731}
\chead{HW 2}
\cfoot{\thepage}

\begin{document}
\title{Homework 2}
\author{Tim Vigers}
\date{\today}
\maketitle

\section{BD 1.1.1}
\begin{enumerate}
  \item Example (a)
  \begin{enumerate}
     \item Here let $X$ be a R.V. indicating the diameter of a pebble and $Y=log(X)$. The logarithm of the diameter is normally distributed, so: $$P_Y(Y)=\frac{1}{\sqrt{2\pi}\sigma}e^{-\frac{1}{2}(\frac{y-\mu}{\sigma})^2}$$
     To find the distribution of $X$, we can do a simple transformation using $\frac{d}{dx}Y=\frac{1}{X}$ and see that $$P_X(X)=\frac{1}{x\sqrt{2\pi}\sigma}e^{-\frac{1}{2}(\frac{log(x)-\mu}{\sigma})^2}$$
     \item Pebble diameters must be $X\in (0,\infty)$, so $-\infty<log(X)<\infty$. Because we are assuming $log(X)\sim \mathcal{N}(\mu,\sigma^2)$, $-\infty<\mu<\infty$ and $\sigma>0$.
     \item This is a parametric model because we are assuming a distribution for the pebble diameters.
   \end{enumerate}
  \item Example (b)
  \begin{enumerate}
     \item For this example we have the model $X_i=\mu+\epsilon_i$, for $1\leq i \leq n$ and $\epsilon\sim \mathcal{N}(0.1,\sigma^2)$. Therefore $$P_X(X)=\frac{1}{\sqrt{2\pi}\sigma}e^{-\frac{1}{2}(\frac{x-\mu+0.1}{\sigma})^2}$$
     \item In this case the variance of the errors is known, so the parameter space is $\mu\in R$.
     \item This is also a parametric model because we are assuming a distribution for the errors.
   \end{enumerate}
   \item Example (c)
   \begin{enumerate}
      \item
      \item
      \item
    \end{enumerate}
\end{enumerate}
\subsection{1.1.2}
\end{document}
